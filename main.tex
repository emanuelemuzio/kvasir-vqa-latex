\documentclass[a4paper,12pt]{report}
\usepackage{unipa-thesis}
\usepackage{subfiles}
\usepackage{float}
\usepackage{gensymb}
\usepackage{amsmath}
\usepackage{textcomp}
\usepackage{glossaries}
\usepackage[utf8]{inputenc}
\usepackage{array}
\usepackage{listings}
\usepackage{url}
\usepackage{graphicx}
\usepackage[export]{adjustbox}

\renewcommand{\schooltitle}{Dipartimento di Ingegneria} 
\renewcommand{\coursename}{Ingegneria Informatica - LM 32}
\renewcommand{\departmentname}{Curriculum Intelligenza Artificiale}
\renewcommand{\thesistitle}{Tecniche sperimentali per i Vision-Language Models applicate al KvasirVQA}
\renewcommand{\authorname}{Emanuele Muzio}
\renewcommand{\advisorname}{Marco La Cascia}
\renewcommand{\academicyear}{2023 - 2024}
\renewcommand{\contentsname}{Indice}
\renewcommand{\chaptername}{Capitolo}
\renewcommand{\bibname}{Bibliografia}

\begin{document}

\titlepagecontent

\tableofcontents

\chapter{Introduzione}
\subfile{chapters/1_introduzione}

\chapter{Dataset}
\subfile{chapters/2_dataset}

\chapter{Stato dell'arte}
\subfile{chapters/3_sota}

\chapter{Approccio Proposto}
\subfile{chapters/4_approccio-proposto}

\chapter{Esperimenti}
\subfile{chapters/5_esperimenti}

\chapter{Conclusione}
\subfile{chapters/6_conclusione}

\chapter{Glossario}
\subfile{glossario}

\bibliographystyle{plain}
\bibliography{references}

\end{document}
