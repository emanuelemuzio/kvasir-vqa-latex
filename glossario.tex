\documentclass[main.tex]{subfiles}
\usepackage[utf8]{inputenc}
\usepackage{glossaries}

\begin{document}

\paragraph{CNN - Convolutional Neural Network} 
Reti neurali progettate per analizzare dati visivi, utilizzando filtri convoluzionali per estrarre caratteristiche rilevanti.

\paragraph{CoT - Chain of Thought} 
Tecnica di prompting che guida il modello a generare ragionamenti espliciti e passo-passo per risolvere problemi complessi.

\paragraph{CV - Computer Vision} 
Campo dell'intelligenza artificiale che si occupa dell'elaborazione e comprensione automatica di immagini e video.

\paragraph{KB - Knowledge Base} 
Archivio strutturato di conoscenze utilizzato per supportare processi di ragionamento automatico e rispondere a domande basate su fatti.

\paragraph{LLM - Large Language Model} 
Modello linguistico di grandi dimensioni, progettato per elaborare e generare testo con capacità avanzate di comprensione e generazione del linguaggio naturale.

\paragraph{LSTM - Long Short-Term Memory} 
Una variante delle reti neurali ricorrenti progettata per apprendere dipendenze a lungo termine nei dati sequenziali, evitando il problema del gradiente evanescente.

\paragraph{MSCOCO - Microsoft Common Objects in Context} 
Dataset di immagini annotato per compiti come object detection, segmentation e captioning, noto per la sua varietà e complessità.

\paragraph{NLP - Natural Language Processing} 
Campo dell'intelligenza artificiale che si occupa dell'elaborazione e comprensione automatica del linguaggio naturale.

\paragraph{OCR - Optical Character Recognition} 
Tecnologia che consente di estrarre e digitalizzare automaticamente il testo contenuto in immagini o documenti scansionati.

\paragraph{RPN - Region Proposal Network} 
Rete neurale utilizzata per generare proposte di regioni di interesse in un'immagine, comunemente impiegata in compiti di detection degli oggetti.

\paragraph{VLM - Vision-Language Model} 
Modello multimodale progettato per elaborare e comprendere congiuntamente dati visivi e testuali.

\paragraph{VQA - Visual Question Answering} 
Task multimodale che combina l'elaborazione di immagini e testo per rispondere a domande relative al contenuto visivo.

\end{document}